% AppendixA
\newpage
\newpage
\chapter{Glossario} % Main appendix title
\label{AppendixA} % For referencing this appendix elsewhere, use \ref{AppendixA}

\lhead{\emph{Glossario}} % This is for the header on each page - perhaps a shortened title

\textbf{AdaBoost:} meta-algoritmo formulato da Yoav Freund ed Robert Schapire, basato sull'idea di creare una regola di predittività altamente accurata combinando regole relativamente deboli con regole inaccurate.

\textbf{Ant:} libreria Java il cui compito è quello di guidare il processo di build di applicazioni Java. Fornisce numerose funzionalità in grado di di compilare, assemblare, testare ed avviare le applicazioni.

\textbf{API:} insieme di procedure disponibili al programmatore, di solito raggruppate a formare un set di strumenti specifici per l'espletamento di un determinato compito all'interno di un certo programma. Spesso con tale termine si intendono le librerie software disponibili in un certo linguaggio di programmazione.

\textbf{C++:} linguaggio di programmazione orientato agli oggetti, con tipizzazione statica. Sviluppato da Bjarne Stroustrup nel 1983 come un miglioramento del linguaggio C. Tra i miglioramenti principali troviamo: l'introduzione del paradigma di programmazione a oggetti, funzioni virtuali, overloading degli operatori, ereditarietà multipla, template e gestione delle eccezioni.

\textbf{Dalvik VM:} è una macchina virtuale, progettata da Dan Bornstein. È ottimizzata per sfruttare la poca memoria presente nei dispositivi mobili, consente di far girare diverse istanze della macchina virtuale contemporaneamente e nasconde al sistema operativo sottostante la gestione della memoria e dei thread. Dalvik è spesso associato alla macchina virtuale Java, anche se il bytecode con cui lavora non è Java. Altre differenze con la JVM tradizionale sono la mancata gestione delle eccezioni e l'architettura a registri.

\textbf{Diaframma:} In fotografia ed in ottica, un diaframma è un'apertura solitamente circolare o poligonale, incorporata nel barilotto dell'obiettivo, che ha il compito di controllare la quantità di luce che raggiunge la pellicola (in una macchina fotografica analogica) o i sensori (in una macchina fotografica digitale) nel tempo in cui l'otturatore resta aperto (tempo di esposizione).

\textbf{Gradle:} recente strumento di build che Google vuole portare a standard per Android. Caratteristiche ne sono la maggiore estensibilità ed utilizzabilità rispetto ad Ant.

\textbf{IDE:} software che consiste di più componenti: editor di codice sorgente, un compilatore e/o un interprete, un tool di building automatico, (solitamente) un debugger. Può essere integrato anche con un sistema di controllo di versione e con uno o più tool per semplificare la costruzione di una GUI.

\textbf{ISO:} velocità della pellicola, detta anche sensibilità o rapidità, indica la sensibilità di una pellicola fotografica (o del sensore in una fotocamera digitale) alla luce.

\textbf{JNI:} framework del linguaggio Java che consente al codice Java di richiamare (o essere richiamato da) codice cosiddetto "nativo", ovvero specifico di un determinato sistema operativo o, più in generale, scritto in altri linguaggi di programmazione, in particolare C, C++ ed Assembly.

\textbf{Maven:} ospitato da Apache Software Foundation, è un software usato principalmente per la gestione di progetti Java e build automation. Per funzionalità è similare ad Apache Ant, ma basato su concetti differenti. Può essere usato anche in progetti scritti in C\#, Ruby, Scala e altri linguaggi.

\textbf{QEMU:} creato da Fabrice Bellard, è un software che implementa un particolare sistema di emulazione che permette di ottenere un'architettura informatica nuova e disgiunta in un'altra che si occuperà di ospitarla. 

