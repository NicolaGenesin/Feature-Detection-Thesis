%%%%%%%%%%%%%%%%%%%%%%%%%%%%%%%%%%%%%%%%%
% Masters/Doctoral Thesis 
% LaTeX Template
% Version 1.42 (19/1/14)
%
% This template has been downloaded from:
% http://www.latextemplates.com
%
% Original authors:
% Steven Gunn 
% http://users.ecs.soton.ac.uk/srg/softwaretools/document/templates/
% and
% Sunil Patel
% http://www.sunilpatel.co.uk/thesis-template/
%
% License:
% CC BY-NC-SA 3.0 (http://creativecommons.org/licenses/by-nc-sa/3.0/)
%
% Note:
% Make sure to edit document variables in the Thesis.cls file
%
%%%%%%%%%%%%%%%%%%%%%%%%%%%%%%%%%%%%%%%%%

%----------------------------------------------------------------------------------------
%	PACKAGES AND OTHER DOCUMENT CONFIGURATIONS
%----------------------------------------------------------------------------------------

\documentclass[10pt, a4paper, oneside]{Thesis} % Paper size, default font size and one-sided paper

\graphicspath{{Pictures/}} % Specifies the directory where pictures are stored

\usepackage[square, numbers, comma, sort&compress]{natbib} % Use the natbib reference package - read up on this to edit the reference style; if you want text (e.g. Smith et al., 2012) for the in-text references (instead of numbers), remove 'numbers' 
\usepackage{setspace}
\singlespacing
%\hypersetup{urlcolor=red, colorlinks=true} % Colors hyperlinks in blue - change to black if annoying
\title{\ttitle} % Defines the thesis title - don't touch this

\begin{document}

%\frontmatter % Use roman page numbering style (i, ii, iii, iv...) for the pre-content pages

\setstretch{1.0} % Line spacing of 1.3

% Define the page headers using the FancyHdr package and set up for one-sided printing
\fancyhead{} % Clears all page headers and footers
\rhead{\thepage} % Sets the right side header to show the page number
\lhead{} % Clears the left side page header

\pagestyle{fancy} % Finally, use the "fancy" page style to implement the FancyHdr headers

\newcommand{\HRule}{\rule{\linewidth}{0.5mm}} % New command to make the lines in the title page

% PDF meta-data
\hypersetup{pdftitle={\ttitle}}
\hypersetup{pdfsubject=\subjectname}
\hypersetup{pdfauthor=\authornames}
\hypersetup{pdfkeywords=\keywordnames}

%----------------------------------------------------------------------------------------
%	TITLE PAGE
%----------------------------------------------------------------------------------------

\begin{titlepage}
\begin{center}


\vspace*{10 mm}
\textsc{\LARGE Università degli studi di Padova}\\[0.5cm] % University name
\textsc{\Large Dipartimento di Matematica \\
Corso di Laurea in Informatica}\\[1.0cm] % Thesis type

\includegraphics[scale=0.15]{../Figures/logo_uni_bw.png}\\[1.5cm]

%\HRule \\[0.4cm] % Horizontal line
{\huge \bfseries Face Detection Engine su piattaforma Android}\\[2.5cm] % Thesis title
%\HRule \\[1.5cm] % Horizontal line
 
\begin{minipage}{0.4\textwidth}
\begin{flushleft} \large
\emph{Laureando:}\\
{Nicola GENESIN \\ 1008671} % Author name - remove the \href bracket to remove the link
\end{flushleft}
\end{minipage}
\begin{minipage}{0.4\textwidth}
\begin{flushright} \large
\emph{Relatore:} \\
{Professor Claudio Enrico PALAZZI} % Supervisor name - remove the \href bracket to remove the link  
\end{flushright}
\end{minipage}\\[3cm]
 
{\large Anno accademico 2013/2014}\\[4cm] % Date

 
\vfill
\end{center}

\end{titlepage}





\clearpage
\newpage
\thispagestyle{empty}
\mbox{}
\clearpage

\vspace*{7cm}
\begin{centering}
\textbf{Sommario}\\


La presente relazione ha come scopo la descrizione dell'attivita di stage, svolta dal sottoscritto,
nel periodo ottobre 2013 - gennaio 2014 presso l'azienda SanMarco Informatica. Il primo capitolo descrive l'azienda
ospitante. Il secondo capitolo espone le motivazioni e gli obiettivi del progetto di stage. Il terzo
capitolo illustra in modo approfondito le attivita effettuate per raggiungere gli obiettivi prefissati.
Il quarto ed ultimo capitolo riporta una valutazione a posteriori sul lavoro svolto, sulle conoscenze
acquisite e sulla distanza tra le conoscenze richieste e le conoscenze possedute.
\end{centering}

\clearpage
\textbf{Convenzioni tipografiche}\\


Al fine di migliorare la leggibilità e la chiarezza dei contenuti esposti sono state adottate le seguenti
convenzioni tipografiche:

\begin{itemize}
  \item \textbf{\textit{Grassetto corsivo}} per i termini ritenuti di comprensione non immediata (tali termini saranno riportati nel Glossario)
  \item \textit{Corsivo} per i termini in lingua non italiana
  \item \textit{Corsivo} per i termini che individuano tecnologie
\end{itemize}

\clearpage % Start a new page

%----------------------------------------------------------------------------------------
%	LIST OF CONTENTS/FIGURES/TABLES PAGES
%----------------------------------------------------------------------------------------

\pagestyle{fancy} % The page style headers have been "empty" all this time, now use the "fancy" headers as defined before to bring them back

\lhead{\emph{Contents}} % Set the left side page header to "Contents"
\tableofcontents % Write out the Table of Contents

\lhead{\emph{List of Figures}} % Set the left side page header to "List of Figures"
\listoffigures % Write out the List of Figures

\lhead{\emph{List of Tables}} % Set the left side page header to "List of Tables"
\listoftables % Write out the List of Tables

%----------------------------------------------------------------------------------------
%	DEDICATION
%----------------------------------------------------------------------------------------

\setstretch{1.3} % Return the line spacing back to 1.3

\pagestyle{empty} % Page style needs to be empty for this page

\dedicatory{Dedicato a} % Dedication text

\addtocontents{toc}{\vspace{2em}} % Add a gap in the Contents, for aesthetics

%----------------------------------------------------------------------------------------
%	THESIS CONTENT - CHAPTERS
%----------------------------------------------------------------------------------------

\mainmatter % Begin numeric (1,2,3...) page numbering

\pagestyle{fancy} % Return the page headers back to the "fancy" style

% Include the chapters of the thesis as separate files from the Chapters folder
% Uncomment the lines as you write the chapters

% Chapter 1

\chapter{Dominio applicativo} % Main chapter title

\label{Chapter1} % For referencing the chapter elsewhere, use \ref{Chapter1} 

\lhead{Capitolo 1. \emph{Dominio Applicativo}} % This is for the header on each page - perhaps a shortened title

La prima parte di questa relazione vuole presentare al lettore il contesto di lavoro dell'azienda ospitante il progetto di stage.

%----------------------------------------------------------------------------------------

\section{Azienda}

\begin{figure}[h]\centering  
\includegraphics[scale=0.38]{../Figures/logo_sanmarco.png}
\caption[Long caption]{Caption}
\label{pic-a}
\end{figure}

Nata negli anni '80 come software house specializzata negli applicativi per aziende manifatturiere, SanMarco Informatica (http://www.sanmarcoinformatica.it/) è una società italiana che mira principalmente a:


%----------------------------------------------------------------------------------------

\section{Prodotti e clienti}

%----------------------------------------------------------------------------------------

\section{Tecnologie}

%----------------------------------------------------------------------------------------



%----------------------------------------------------------------------------------------


% Chapter 1

\chapter{Progetto Aziendale} % Main chapter title

\label{Chapter2} % For referencing the chapter elsewhere, use \ref{Chapter1} 

\lhead{Capitolo 2. \emph{Progetto Aziendale}} % This is for the header on each page - perhaps a shortened title

%----------------------------------------------------------------------------------------

\section{Introduzione al riconoscimento facciale}

Il riconoscimento facciale è una tecnica di intelligenza artificiale utilizzata per identificare o verificare l'identità di una persona a partire da una o più immagini che la ritraggono.\\
Si può affermare che si tratti di un riconoscimento di pattern (dove il pattern da riconoscere è un volto umano), che avviene mediante tecniche di elaborazione di immagini digitali.\\Vi sono vari sistemi che adempiono a tali risultati, suddivisibili in due categorie:

\begin{itemize}
  \item \textbf{Semplici}: partono dal presupposto che un viso sia composto da tutti gli elementi comuni tipici della nostra specie (occhi, naso ,bocca, orecchie, etc.). Costo computazionale ridotto, funzionano efficacemente nel caso il volto sia frontale rispetto all'immagine digitale
  \item \textbf{Complessi}: sommano al sistema semplice algoritmi che riescono ad eseguire riconoscimento facciale anche nel caso il volto sia ruotato o comunque non in visione frontale. Costo computazionale generalmente molto più elevato. 
\end{itemize}

Data una singola immagine digitale, i maggiori problemi che si incontrano nella ricerca della presenza di volti sono:

\begin{itemize}
\item Qualità dell'immagine non adeguata (sgranata, non a fuoco, risoluzione insufficiente)
\item Condizione d'illuminazione non ottimale (sottoesposizione o sovraesposizione eccessiva)
\item Espressioni facciali non campionate (quando i volti presenti assumono espressioni non presenti nel template di confronto)
\item Variazione della "messa in posa"
\item Variazioni d'aspetto del volto (ad esempio i tratti di un volto di una persona di origine asiatica saranno notevolmente diversi da quelli di una persona di origine europea)
\end{itemize}

\subsection{Machine Learning}

Data la complessità del problema è difficile creare una soluzione unica che abbia un'accuratezza soddisfacente. Una delle vie migliori per costruire un classificatore è collezionare una grande database di esempi di classificazione, creare una funzione di classificazione parametrizzata e successivamente ottimizzarla in modo tale da avere degli output che coincidano al meglio con gli esempi. Questa metodologia è definita machine learning

%----------------------------------------------------------------------------------------

\section{Obiettivi}

L'esigenza dell'azienda ospitante era quello di creare un core engine di riconoscimento facciale che potesse funzionare su dispositivi Android. Questo senza inficiare eccessivamente le prestazioni dell'applicazione principale. Tale esigenza era dettata dal fatto di voler applicare tale funzionalità ad una famiglia di prodotti molto diversificata come ad esempio:

\begin{itemize}
\item \textbf{Oculistica} Applicazione rivolta direttamente ai clienti dove questi ultimi avrebbero potuto provare gli occhiali (overlay del modello 3D dell'occhiale) selezionandoli direttamente dal dispositivo mobile
\item \textbf{Intrattenimento}
\begin{itemize}
\item \textbf{Videogiochi} 
utilizzo del riconoscimento facciale per variare l'esperienza d'utilizzo del videogioco a seconda della posizione e della quantità di volti presenti di fronte al dispositivo
\item \textbf{Generico}
utilizzo del riconoscimento facciale per applicazioni fun generiche come il face-swapping 
\end{itemize}
\end{itemize}  

Oltre a tale obiettivo puramente tecnico è stata richiesta l'integrazione verso i social network per la condivisione di foto o di altri contenuti multimediali.

%----------------------------------------------------------------------------------------

\section{Aspettative}

L'azienda si aspettava che l'obiettivo principale venisse raggiunto.

%----------------------------------------------------------------------------------------

\section{Vincoli}

Non sono stati imposti vincoli particolari per quanto riguarda lo svolgimento del progetto di stage.
I vincoli imposti sono:
\begin{itemize}
 	\item Utilizzare il linguaggio Java per l'implementazione del prototipo
 	\item Utilizzare Eclipse e nello specifico l'SDK Android che fornisce le librerie API e gli strumenti di sviluppo necessarie al build, test e debug delle applicazioni
 	\item Garantire flessibilità ed estensibilità futura del prototipo
 	\item Garantire il funzionamento per le versioni Android uguali o superiori alla v2.3 (Gingerbread)
 	\item Fornire documentazione adeguata sul prototipo implementato
 	\item Rispettare la timeline assegnata
\end{itemize}


%----------------------------------------------------------------------------------------

 
% Chapter 1

\chapter{Attività di Stage} % Main chapter title

\label{Chapter3} % For referencing the chapter elsewhere, use \ref{Chapter1} 

\lhead{Capitolo 3. \emph{Attività di stage}} % This is for the header on each page - perhaps a shortened title

%----------------------------------------------------------------------------------------

\section{Pianificazione}

Viene di seguito riportato un diagramma di Gantt riassuntivo della pianificazione
del lavoro nel periodo di stage. Come si può vedere le ore sono state suddivise secondo
2 obiettivi, che rappresentano parti di sistema indipendenti che verranno sviluppate
in maniera autonoma.

GRAFICO

Il modello di ciclo di vita prescelto può essere paragonabile allo scrum: viene infatti preferito rispetto alla rigidità dei modelli più classici, in
quanto permetteva lo sviluppo continuo dei vari incrementi e la possibilità di mostrare
passo per passo i risultati ottenuti. Per chiarezza si riporta una breve descrizione degli
obiettivi presenti nel piano di lavoro che verranno maggiormente approfonditi nelle
sezioni successive:

\textbf{Obiettivo 1}: Creazione di una libreria che permetta il riconoscimento facciale\\
\textbf{Obiettivo 2}: Creazione di una libreria che permetta una facile condivisione dei contenuti multimediali eventualmente creati verso i social network


%----------------------------------------------------------------------------------------

\section{Analisi}

In seguito allo studio del dominio applicativo effettuato durante le prime ? settimane di stage e alle spiegazioni del tutor aziendale sono stati individuati i requisiti del prototipo richiesto.
Dal momento che si tratta di un prototipo che dovrà essere alla base di una famiglia di prodotti, esso non sarà indipendente dall input umano ma dovrà garantire la flessibilità ed espandibilità necessaria.

Di seguito vengono riportati i casi d'uso principali individuati.

\subsection{Casi d'uso}

Il caso d'uso è un tecnica usata nei processi di ingegneria del software per effettuare in maniera esaustiva e non ambigua, la raccolta dei requisiti funzionali al fine di produrre software di qualità.
Essa consiste nel valutare ogni requisito focalizzandosi sugli attori che interagiscono col sistema, valutandone le varie interazioni. Nel caso specifico, l'attore
principale è uno solo, ossia l'utente che utilizza il dispositivo mobile.

\textbf{UC1 Caso d'uso generale}
\\
\textbf{Precondizione:}
\\
\textbf{Postcondizione:}
\\
\textbf{Scenario principale:}
\\
GRAFICO UML
\\


\textbf{UC2 Caso d'uso generale}
\\
\textbf{Precondizione:}
\\
\textbf{Postcondizione:}
\\
\textbf{Scenario principale:}
\\
GRAFICO UML
\\


\textbf{UCx Caso d'uso generale}
\\
\textbf{Precondizione:}
\\
\textbf{Postcondizione:}
\\
\textbf{Scenario principale:}
\\
GRAFICO UML
\\

\subsection{Requisiti}



La classificazione dei requisiti sarà seguirà il formalismo:
\begin{center}
	\textbf{R\{TIPO\}\{RICHIESTA\}\{GERARCHIA\}}
\end{center}

Nello specifico:
\begin{itemize}
\item TIPO può essere:
	\begin{itemize}
		\item F, indica un requisito funzionale
		\item Q, indica un requisito di qualità
		\item V, indica un requisito di vincolo
	\end{itemize} 
\item IMPEGNO può essere:
	\begin{itemize}
		\item O, indica un requisito obbligatorio
		\item D, indica un requisito desiderabile
	\end{itemize} 
\item GERARCHIA: i requisiti sono organizzati gerarchicamente secondo una struttura ad albero. Se la complessità di un requisito viene ritenuta elevata, questo può essere diviso in sotto-requisiti.
\end{itemize}

A seguire i requisiti individuati:

\begin{center}
    \begin{tabular}{ | p{3cm} | l | p{3cm} |}
    \hline
    Requisiti & Descrizione & Caso d'uso \\ \hline
    test & test & test  \\ \hline 
    test & test & test \\ \hline
    test & test & test  \\ \hline 
    test & test & test \\ \hline
    test & test & test  \\ \hline 
    test & test & test \\ \hline
    test & test & test  \\ \hline 
    test & test & test \\ \hline
    \end{tabular}
\end{center}

\subsubsection{Modello di ciclo di vita}
%----------------------------------------------------------------------------------------

\section{Progettazione}

\subsection{Architettura}
\subsection{Tecnologie utilizzate}

\subsubsection{Java}

Il linguaggio per applicazioni Android è in realtà un "dialetto" del linguaggio Java così come è diversa anche la virtual machine di runtime (Dalvik virtual machine anziché JVM).
In una normale applicazione Android non c'è un entry point (il classico metodo "main") da dove normalmente un programma comincia a caricare le sue parti software e avviarsi: tutto è pensato per essere un "componente" pilotato dagli eventi ("Event Driven") dell'hardware o di altri componenti. Questo paradigma fa sì che il programmatore sviluppi per ogni hardware delle routine il più possibile indipendenti. Un vantaggio è che il sistema operativo potrà ottimizzare le risorse, ad esempio rinunciando a caricare componenti (e hardware) non supportati o non prioritari perché inutilizzati.


\subsubsection{Android NDK}

L NDK (Native Development Kit) è uno strumento che permette di implementare parte dell'applicazione usando codice scritto in linguaggio nativo come il C ed il C++. Per alcune applicazioni può essere utile in quanto si può riutilizzare le librerie scritte in tali linguaggi. L'uso di tale strumento non da però beneficio per la maggior parte delle applicazioni ed è da usare quindi con parsimonia (spesso l'aumento delle performance dovuto all'utilizzo di codice nativo non giustifica l'aumento di complessità). Tipicamente buoni candidati sono programmi che sono CPU-intensive che non allocano molta memoria come simulazioni fisiche ed analisi dei segnali. Nel caso specifico sono stati effettuati dei test per verificare se la quantità di chiamate verso la libreria C/C++ OpenCV giustifica tale strumento. 

\subsubsection{Android SDK}

Le applicazioni di Android sono sviluppate all'interno di un framework, ossia di una struttura dati specifica. La struttura del framework è molto chiara se si utilizza l'ambiente di sviluppo (Android SDK) con Eclipse; il mancato utilizzo di Eclipse, tuttavia, non impedisce di scrivere applicazioni Android funzionanti.
Le applicazioni Android sono caratterizzate da una certa dualità: parti dinamiche scritte in Java e parti statiche scritte in XML. Tipico delle parti statiche possono essere quelle caratteristiche che non cambiano durante l'esecuzione dell'applicazione, come per esempio il colore dello sfondo. Tipico delle parti dinamiche sono invece gli aspetti programmatici come per esempio la gestione degli eventi.
Questa dualità è però solo apparente. Durante l'esecuzione, infatti, l'ambiente di esecuzione o run-time noto come Dalvik virtual machine (DVM), che in tale ambito sostituisce la consueta Macchina virtuale Java (JVM), esegue sempre un programma. Per lo sviluppo delle applicazioni è disponibile una completa documentazione[98] la quale, anche graficamente, riprende la struttura tipica della documentazione Java[99] del sito Oracle.

Tramite l'SDK possiamo passare dalla descrizione della nostra applicazione alla sua effettiva esecuzione sia in emulazione, sia su un dispositivo concreto. Per descrivere l'applicazione al dispositivo prescelto si utilizza il file Manifest.xml. Possiamo quindi affermare che un'applicazione è descritta completamente da una tripletta:

\begin{itemize}
\item Codice Java
\item Risorse statiche xml
\item Manifest.xml
\end{itemize}

Il codice Java viene poi compilato insieme all'XML per generare un file con estensione .apk: esso contiene il bytecode per la cosiddetta Dalvik Virtual Machine (DVM). I passi successivi servono per installare il bytecode nel dispositivo (ed eseguirlo in emulazione).



\subsubsection{Android FD-Library}

All'interno delle SDK Android è presente una API per il riconoscimento facciale: Android.Media.FaceDetector. Questa classe permette, data un immagine, di trovare le eventuali facce presenti attraverso il semplice utilizzo del metodo findFaces(). Ogni istanza ritornata viene salvata in un array Faces[] e contiene:

\begin{itemize}
\item Valutazione se si tratta di un volto o meno 
\item Distanza tra gli occhi (numero di pixel)
\item Posizione (x,y) del punto medio situato tra i due occhi
\item Rotazioni di posa della faccia (x,y,z)
\end{itemize} 

A seguito di valutazioni e di un prototipo usa e getta si è però deciso di non utilizzare in quanto:

\begin{itemize}
\item Non ritorna il rettangolo che include con precisione la faccia

\item Non consente di utilizzare template a piacimento per la ricerca di feature specializzate (naso, occhi, orecchie, etc.)

\item Prestazioni discrete per una immagine unica, ma non adatta ad un applicazione real-time

\item Supporta solo bitmap in formato RGB\_565 

\item Sono stati rilevati comportamenti evidentemente diversi a seconda del dispositivo utilizzato, e la quantità di bug riportata dalla community Android è rilevante
\end{itemize}

\subsubsection{OpenCV}

OpenCV è una libreria creata da Intel a scopi di commerciali e di ricerca. E' una libreria open e gratuita, e pertanto il suo codice può essere utilizzato nella sua interezza o in parte. Al momento della stesura di questo documento la versione più recente disponibile è la v2.4.7 specifica per dispositivi Android. Tale versione fornisce un wrapper Java alle librerie OpenCV che fornisce quasi tutte le fonzionalità core. Al fine di garantire il funzionamento di applicativi basati su questa libreria è però necessario installare sul dispositivo un software chiamato OpenCV Manager, che permette all'applicazione ottimizzazioni in base all'hardware presente.

Vi sono due modi d'utilizzo di questa libreria:

\begin{itemize}
\item \textbf{Ad alto livello}:  utilizzo esclusivo delle Java API. Facilità di sviluppo, ma non permette l'accesso completo alle libreria. Prestazioni minori a causa al gran numero di chiamate JNI (Java Native Interface) verso la libreria

\item \textbf{A basso livello}:  utilizzo dell'interfaccia nativa di openCV. Android permette infatti di eseguire chiamate a funzioni native, ciò significa che è possibile utilizzare l'interfaccia C++ di OpenCV. 
\end{itemize}
%----------------------------------------------------------------------------------------





\subsection{Associazione classi-requisiti}


\section{Implementazione}



%----------------------------------------------------------------------------------------

\section{Verifica e validazione}

\subsection{Analisi Statica}
Al fine di portare avanti il processo di verifica con metodo è stato necessario renderlo
quantificabile. Perciò si sono definite delle metriche sul codice sorgente. Esse sono
di seguito definite e il loro significato viene descritto e spiegato. Per ogni metrica si
sono definiti un range di accettazione e un range ottimale.

\subsubsection{Complessità ciclomatica}

Tale metrica indica il numero di cammini linearmente indipendenti percorribili durante
l'esecuzione di un singolo metodo. Tale metrica è molto importante in quanto ha
implicazioni dirette sulle attività di testing: infatti essa rappresenta un upper bound
per il numero di routine di test necessarie per raggiungere un completo branch coverage.

\textbf{Range dichiarato:}
\begin{itemize}
\item Accettazione $\leq 15$
\item Ottimale $\leq 10$
\end{itemize}

\subsubsection{LOC}

Tale metrica (Lines of Code, linee di codice) indica il numero di linee di codice. Nel conteggio vengono escluse le righe contenenti dichiarazioni di namespaces, tipi, campi, metodi oltre che i metodi astratti e i tipi enumeration.
Solo il codice effettivamente eseguito è considerato nel conteggio delle righe.
La metrica LOC non è sempre legata alla produttività del programmatore, ma torna
utile sia nel calcolo della percentuale di statement coverage e nella valutazione del
software.
Essa è indice di manutenibilià (è più semplice manutenere metodi brevi) e quindi
qualità del codice.

\textbf{Range per metodo:}
\begin{itemize}
\item Accettazione $\leq 45$
\item Ottimale $\leq 15$
\end{itemize}

\subsubsection{Numero di campi utilizzati per classe}

Indica il numero di membri di classe (campi dati) di una particolare classe. E'
importante in quanto è indice delle responsabilità andate ad una classe, se viene
superato può indicare che la classe ne raggruppa troppe altre e andrebbe ulteriormente
separata in classi distinte.

\textbf{Range dichiarato:}
\begin{itemize}
\item Accettazione $\leq 16$
\item Ottimale $\leq 8$
\end{itemize}

\subsubsection{Numero di metodi utilizzati per classe}

Indica il numero di metodi di una particolare classe. Similmente al numero di campi
per classe è importante in quanto è indice delle responsabilità andate ad una classe, se
viene superato può indicare che la classe ha troppe funzionalità e andrebbe separata.

\textbf{Range dichiarato:}
\begin{itemize}
\item Accettazione $\leq 16$
\item Ottimale $\leq 8$
\end{itemize}

\subsubsection{LCOM e LCOM HS}

Si vuole definire una metrica a partire da questa definizione: se una classe ha una sola responsabilità si
dice che essa è coesa. In generale una classe risulterà coesa se i suoi metodi sono
strettamente legati fra loro. Quando metodi distinti non usano attributi o metodi
comuni significa che essi non condividono nulla e che quindi potrebbero venir separati.
La metrica di LCOM misura quindi quanto poco una classe è coesa. Alcune formule per il
calcolo di LCOM sono le seguenti:

\begin{center}
\textbf{LCOM} =  $1-(\sum(MF)/M \ast F) $  \\
\textbf{LCOM HS} = $(M-(\sum(MF)/F))\ast(M-1)$
\end{center}

Dove:
\begin{itemize}
\item M è il numero di metodi nella classe. Si considerano sia i metodi statici sia quelli di istanza, ed include inoltre i costruttori,getters e setters ed eventuali metodi del tipo add/remove
\item F è il numero di campi dati d'istanza all'interno della classe 
\item MF è il numero di metodi della classe che accedono un particolare campo dati di tipo classe
\item $\sum(MF)$ è la somma degli MF tra tutti i campi d'istanza della classe
\end{itemize}

L'idea alla base di tali formule è che una classe perfettamente coesa utilizza tutti i
suoi campi dati di tipo classe all'interno di ogni suo metodo il che comporta :

\begin{center}
 $(\sum(MF) = M \ast F) \Leftrightarrow LCOM = 0 $
\end{center}



La differenza tra LCOM e la sua versione HS (Hendersons-Sellers) è che la prima
ritorna dei valori nel range [0-1] mentre la seconda nel range [1-2]. La versione HS è
considerata più efficace per la rilevazione dei tipi scarsamente coesi.
Tale metrica è interessante, ma va trattata con cautela (basti pensare una classe
che ha n campi dati, n getters e n setters, essa risulterà scarsamente coesa, il che
non rispecchia la realtà). Infatti essa non va valutata singolarmente, ma deve essere
inserita in un contesto di valutazione che comprende altre variabili, soprattutto il
numero di campi e il numero di metodi. 

\textbf{Range dichiarato:}
\begin{itemize}
\item Accettazione $\leq 0.8$
\item Ottimale = 0
\end{itemize}


\subsection{Analisi Dinamica}
\subsection{Test di Validazione}

\section{Esiti attività}

\section{Valutazione prestazioni}

Al fine di verificare le prestazioni ottenute in media dall'applicativo, l'azienda ha fornito 4 dispositivi che rappresentano all'anno della stesura di questo documento (2014) le possibili fasce di mercato.

I dispositivi forniti sono:

\begin{itemize}
\item \textbf{Fascia Alta - \textit{Galaxy S4}} 
	\begin{itemize}
		\item CPU quad core 1.6GHz Cortex A15 
		\item RAM 2 GB 
		\item Fotocamera : posteriore: 13 MP , anteriore 2.1 MP
		\item Android 4.2.2 Jelly Bean 
	\end{itemize}
\item \textbf{Fascia Media - \textit{Galaxy S3 Mini}}
	\begin{itemize}
		\item CPU 1 GHz dual core ARM Cortex A9
		\item RAM 1 GB 
		\item Fotocamera : posteriore: 5 MP , anteriore 1 MP
		\item Android 4.1.1 Jelly Bean 
	\end{itemize}
\item \textbf{Fascia Bassa - \textit{Galaxy S}}
	\begin{itemize}
		\item CPU 0.8 GHz single core ARM Cortex A9
		\item RAM 1 GB 
		\item Fotocamera : anteriore 0.8 MP
		\item Android 2.3.0 Gingerbread
	\end{itemize}
\end{itemize}

Dal momento che le prestazioni possono dipendere dalla qualità della luce e dalla distanza dell utente dal dispositivo, si è preferito testare l'applicazione utilizzando video come input. Le variabili da tenere in considerazione sono risoluzione, numero di volti , numero di falsi positivi, numero di fotogrammi medi per secondo al termine di ogni video. I video sono sono volutamente caratterizzati da camere mobili, e da scene in cui non sono presenti persone al fine di verifica la presenza di falsi positivi.

I video utilizzati sono reperibili direttamente dal web ai link:

\begin{itemize}
\item[•] \href{http://www.youtube.com/watch?v=IIdGxR-aU6o}{Video 1}
\item[•] \href{http://www.youtube.com/watch?v=yWPyRSURYFQ}{Video 2}
\item[•] \href{http://www.youtube.com/watch?v=sL7cqpIvvRk}{Video 3}
\end{itemize}

Risoluzioni utilizzate:

\begin{itemize}
\item[•] $1280\ast960$ 
\item[•] $640\ast480$ 
\item[•] $320\ast240$ 
\end{itemize}

I seguenti test sono stati effettuati con parametri i seguenti parametri:
\begin{itemize}
\item \textit{FdActivity::mMinNeighbors} = 3 (sono necessari almeno 3 hit per frame per confermare positivamente il volto)
\item \textit{FdActivity::mMinSize} = -1  
\item \textit{FdActivity::mMaxSize} = -1
\item \textit{FdActivity::mScaleFactor} = 1.0 
\end{itemize}

In questo modo ci si assicura che il costo computazionale per ogni frame sia massimo (il metodo di riconoscimento facciale terminerà solo dopo aver valutato l'immagine nella sua interezza). 
Dato che per ogni singolo frame vi possono essere più hits corrispondenti allo stesso volto, è stata creata una semplice funzione ad hoc che permetta di contare un solo volto per set coincidenti di hits. 

\begin{figure}[h]\centering  
\includegraphics[scale=0.6]{../Figures/graph_1.png}
\caption[Long caption]{Caption}
\label{pic-a}
\end{figure}

Al variare dei dispositivi non vi è alcuna differenza nei risultati (quantità di volti rilevati + $\varepsilon$(falsi positivi rilevati)) se non al variare del parametro \textit{FdActivity::mMinNeighbors}. Infatti,prendendo come risoluzione di riferimento $640 \ast 480$, sono stati ottenuti i seguenti risultati
\\
GRAFICO A RES FISSA CON NUM FALSI POSITIVI E NEGATIVI AL VARIARE DELLA RETENTION 
\\
I risultati sono però stati totalmente differenti nell'utilizzo reale utilizzando la fotocamera posteriore come input dei frame al posto dei video (che garantivano un framerate costante di 24 fotogrammi per secondo). 
\\
GRAFICO LUMINOSITA NORMALE

GRAFICO LUMINOSITA BASSA
\\
Infine si può notare come vi siano grosse discrepanze di prestazioni al variare della fotocamera d'input (Anteriore o Posteriore). Questo è dovuto al fatto che generalmente la fotocamera posteriore è di qualità maggiore (sensore di dimensioni maggiori, diaframma più aperto ed ISO che può raggiungere valori più elevati)
\\
GRAFICO FRONTALE VS POSTERIORE

\subsection{Considerazioni}

A seguito dei test effettuati l'idea è che i dispositivi correntemente presenti sul mercato abbiamo una potenza computazionale più che sufficiente per eseguire tali task con prestazioni discrete e garantiscano un'esperienza d'utilizzo adeguata. Il maggior problema riscontrato è sicuramente la qualità della fotocamera che può condizionare drasticamente le performance se posta in condizione d'utilizzo non ottimale. Il software preinstallato sullo smartphone infatti, in condizioni di bassa luminosità, tende ad alzare il valore ISO (sensibilità del sensore) ed ad utilizzare un diaframma il più aperto possibile. La necessità di miniaturizzare l'hardware fotografico ha però come conseguenza quella di non poter avere sensori e lenti di dimensioni adeguate. Questo si traduce in bassa qualità, e pertanto per ogni singolo frame di un video sarà necessario un tempo di esposizione molto elevato. Lo stream di fotogrammi risulterà quindi caratterizzato da un framerate molto basso, mettendo il processore in attesa attiva del fotogramma da processare. 


%----------------------------------------------------------------------------------------


% Chapter 1

\chapter{Valutazione retrospettiva} % Main chapter title

\label{Chapter4} % For referencing the chapter elsewhere, use \ref{Chapter1} 

\lhead{Capitolo 4. \emph{Valutazione Retrospettiva}} % This is for the header on each page - perhaps a shortened title

%----------------------------------------------------------------------------------------

\section{Copertura dei Requisiti}

Vengono riportate le tabelle con gli esiti dell'attività di sviluppo. E' presente una
tabella per ogni tipologia di requisito (di qualità, funzionale, di vincolo). Per ogni
requisito viene riportato l'esito dello stesso dopo lo sviluppo e l'implementazione del
software. L'esito è stato diviso nelle seguenti categorie:
\begin{itemize}
\item \textbf{Soddisfatto:} Indica che il requisito è stato implementato con successo e quindi
soddisfatto. Colore associato \textcolor{green}{verde}
\item \textbf{Non soddisfatto:} Indica che il requisito non è stato implementato con successo.
Colore associato \textcolor{red}{rosso}
\item \textbf{Incompleto:} Indica che il requisito è stato implementato solo in alcune sue
parti. Colore associato \textcolor{yellow}{giallo}
\end{itemize}

\begin{center}
    \begin{tabular}{ | l | l | l | p{5cm} |}
    \hline
    Codice & Descrizione & Tipologia & Esito \\ \hline
    test & test & test & \textcolor{yellow}{giallo} \\ \hline
    test & test & test & \textcolor{yellow}{giallo} \\ \hline
    test & test & test & \textcolor{yellow}{giallo} \\ \hline
    test & test & test & \textcolor{yellow}{giallo} \\ \hline
    test & test & test & \textcolor{yellow}{giallo} \\ \hline
    test & test & test & \textcolor{yellow}{giallo} \\ \hline
    test & test & test & \textcolor{yellow}{giallo} \\ \hline
    test & test & test & \textcolor{yellow}{giallo} \\ \hline
    

    \hline
    \end{tabular}
\end{center}


%----------------------------------------------------------------------------------------

\section{Conoscenze acquisite}

%----------------------------------------------------------------------------------------

\section{Distanza tra conoscenze richieste e conoscenze possedute}

%----------------------------------------------------------------------------------------

 
%\chapter{Verifica e Validazione} % Main chapter title

\label{Capitolo5} % For referencing the chapter elsewhere, use \ref{Chapter1} 

\lhead{Capitolo 5. \emph{Verifica e Validazione}} % This is for the header on each page - perhaps a shortened title

\section{Analisi Statica}
Al fine di portare avanti il processo di verifica con metodo è stato necessario renderlo
quantificabile. Perciò si sono definite delle metriche sul codice sorgente. Esse sono
di seguito definite e il loro significato viene descritto e spiegato. Per ogni metrica si
sono definiti un range di accettazione e un range ottimale.

\subsection{Complessità ciclomatica}

Tale metrica indica il numero di cammini linearmente indipendenti percorribili durante
l'esecuzione di un singolo metodo. Tale metrica è molto importante in quanto ha
implicazioni dirette sulle attività di testing: infatti essa rappresenta un upper bound
per il numero di routine di test necessarie per raggiungere un completo branch coverage.

\textbf{Range dichiarato:}
\begin{itemize}
\item Accettazione $\leq 15$
\item Ottimale $\leq 10$
\end{itemize}

\subsection{LOC}

Tale metrica (Lines of Code, linee di codice) indica il numero di linee di codice. Nel conteggio vengono escluse le righe contenenti dichiarazioni di namespaces, tipi, campi, metodi oltre che i metodi astratti e i tipi enumeration.
Solo il codice effettivamente eseguito è considerato nel conteggio delle righe.
La metrica LOC non è sempre legata alla produttività del programmatore, ma torna
utile sia nel calcolo della percentuale di statement coverage e nella valutazione del
software.
Essa è indice di manutenibilià (è più semplice manutenere metodi brevi) e quindi
qualità del codice.

\textbf{Range per metodo:}
\begin{itemize}
\item Accettazione $\leq 45$
\item Ottimale $\leq 15$
\end{itemize}

\subsection{Numero di campi utilizzati per classe}

Indica il numero di membri di classe (campi dati) di una particolare classe. E'
importante in quanto è indice delle responsabilità andate ad una classe, se viene
superato può indicare che la classe ne raggruppa troppe altre e andrebbe ulteriormente
separata in classi distinte.

\textbf{Range dichiarato:}
\begin{itemize}
\item Accettazione $\leq 16$
\item Ottimale $\leq 8$
\end{itemize}

\subsection{Numero di metodi utilizzati per classe}

Indica il numero di metodi di una particolare classe. Similmente al numero di campi
per classe è importante in quanto è indice delle responsabilità andate ad una classe, se
viene superato può indicare che la classe ha troppe funzionalità e andrebbe separata.

\textbf{Range dichiarato:}
\begin{itemize}
\item Accettazione $\leq 16$
\item Ottimale $\leq 8$
\end{itemize}

\subsection{LCOM e LCOM HS}

Si vuole definire una metrica a partire da questa definizione: se una classe ha una sola responsabilità si
dice che essa è coesa. In generale una classe risulterà coesa se i suoi metodi sono
strettamente legati fra loro. Quando metodi distinti non usano attributi o metodi
comuni significa che essi non condividono nulla e che quindi potrebbero venir separati.
La metrica di LCOM misura quindi quanto poco una classe è coesa. Alcune formule per il
calcolo di LCOM sono le seguenti:

\begin{center}
\textbf{LCOM} =  $1-(\sum(MF)/M \ast F) $  \\
\textbf{LCOM HS} = $(M-(\sum(MF)/F))\ast(M-1)$
\end{center}

Dove:
\begin{itemize}
\item M è il numero di metodi nella classe. Si considerano sia i metodi statici sia quelli di istanza, ed include inoltre i costruttori,getters e setters ed eventuali metodi del tipo add/remove
\item F è il numero di campi dati d'istanza all'interno della classe 
\item MF è il numero di metodi della classe che accedono un particolare campo dati di tipo classe
\item $\sum(MF)$ è la somma degli MF tra tutti i campi d'istanza della classe
\end{itemize}

L'idea alla base di tali formule è che una classe perfettamente coesa utilizza tutti i
suoi campi dati di tipo classe all'interno di ogni suo metodo il che comporta :

\begin{center}
 $(\sum(MF) = M \ast F) \Leftrightarrow LCOM = 0 $
\end{center}

La differenza tra LCOM e la sua versione HS (Hendersons-Sellers) è che la prima
ritorna dei valori nel range [0-1] mentre la seconda nel range [1-2]. La versione HS è
considerata più efficace per la rilevazione dei tipi scarsamente coesi.
Tale metrica è interessante, ma va trattata con cautela (basti pensare una classe
che ha n campi dati, n getters e n setters, essa risulterà scarsamente coesa, il che
non rispecchia la realtà). Infatti essa non va valutata singolarmente, ma deve essere
inserita in un contesto di valutazione che comprende altre variabili, soprattutto il
numero di campi e il numero di metodi. 

\textbf{Range dichiarato:}
\begin{itemize}
\item Accettazione $\leq 0.8$
\item Ottimale = 0
\end{itemize}

\newpage
\section{Analisi Dinamica}

Nello sviluppo dell'applicazione si sono progettati dei test che precedessero l'implementazione di ogni componente e di ogni classe. I test di integrazione hanno definito come i vari
componenti avrebbero dovuto relazionarsi tra di loro, e per ogni componente sono stati definiti adeguati test di unità che andassero a verificare il corretto funzionamento delle classi interne. Ogni unità e stata testata in
relazione al ruolo che essa avrebbe avuto all'interno del sistema in base al suo "peso".
I test d'unità verificano in gran parte l'integrità degli input inseriti dall'utente e di come questi variano durante l'utilizzo dell'applicativo.

Per l'analisi delle chiamate ai metodi forniti dalle librerie esterne (OpenCV e Facebook SDK) si è cercato di
verificare la correttezza nei limiti di risorse e tempo disponibili. Infatti alcune funzioni fornite da tali librerie sono ad alto livello e comprendono un grafo di chiamate molto esteso
(in particolare chiamate a metodi di OpenCV, come il rilevamento facciale, e a Facebook per quanto riguarda il login e il mantenimento della sessione attiva), e non è stato possibile progettare test in quanto tali operazioni sono
particolarmente complesse e avrebbero richiesto una profonda comprensione della meccanica di tali librerie.

Il tool utilizzato per eseguire i test è stato Android JUnit che fornisce delle classi previste di metodi adibiti alla creazione dei cosiddetti oggetti mock (oggetti simili a quelli reali ma che hanno l'obiettivo di simulare l'oggetto reale in maniera controllata) e metodi che possano aiutare ad aiutare il ciclo di vita di un componente.

\section{Test di Validazione}

Vengono riportati in questa sezione i test che descrivono la strategia utilizzata per la
validazione del prodotto. Si è cercato di tenere un rapporto di 1:1 tra i
requisiti funzionali e i test di validazione che verificavano quella specifica funzionalità.
Per ogni test viene riportato il suo codice gerarchico identificativo. Se tale test risulta sufficientemente grande, esso viene scomposto in sotto-test aventi come indice gerarchico l'indice del padre.

\begin{itemize}
\item \textbf{T1} Il test è atto a verificare il buon funzionamento dell'attività di riconoscimento facciale e delle rispettive parti
\begin{itemize}
\item \textbf{T1.1} Il test verifica il corretto caricamento delle librerie e le adatta in base all'hardware caratterizzante dispositivo
\item \textbf{T1.2} Il test verifica la possibilità di eseguire la calibrazione con le
impostazioni selezionate
\begin{itemize}
\item \textbf{T1.2.1} Il test verifica la possibilità di cambiare camera a run-time
\item \textbf{T1.2.2} Il test verifica la possibilità di cambiare il tipo di elaboratore della camera
\item \textbf{T1.2.3} Il test verifica la possibilità di cambiare la risoluzione del frame in ingresso
\item \textbf{T1.2.4} Il test verifica la possibilità di cambiare il range operativo di campionamento
\item \textbf{T1.2.5} Il test verifica la possibilità di cambiare i parametri operativi di campionamento
\item \textbf{T1.2.6} Il test verifica il corretto funzionamento di modifica delle parti da ricercare a run-time
\item \textbf{T1.2.7} Il test verifica la possibilità di abilitare o disabilitare la modalità alto contrasto e/o monocromatica
\end{itemize}


\item \textbf{T1.4} Il test verifica la possibilità di cambiare lo Sprite corrente
\begin{itemize}
\item \textbf{T1.4.1} Il test deve verificare il corretto caricamento degli Sprites
\item \textbf{T1.4.2} Il test deve verificare la corretta renderizzazione della ListView
\item \textbf{T1.4.3} Il test verifica verifica il corretto utilizzo dello Sprite assegnato
\end{itemize}
\end{itemize}
\item \textbf{T2.1} Il test verifica il corretto caricamento di un'immagine dallo smartphone
\item \textbf{T3} Il test verifica la possibilità di condividere la foto
\begin{itemize}
\item \textbf{T3.1} Il test verifica la possibilità di ridimensionare la foto scattata
\item \textbf{T3.2} Il test verifica la possibilità di condividere la foto inoltrandola verso applicazioni di messaggistica preinstallate nello smartphone
\begin{itemize}
\item \textbf{T3.2.1} Il test verifica il caricamento della lista di applicazioni attraverso le quali è possibile condividere il contenuto
\item \textbf{T3.2.2} Il test verifica l'esito della condivisione
\end{itemize}
\item \textbf{T3.3} Il test verifica la possibilità di eseguire la connessione verso la piattaforma Facebook
\begin{itemize}
\item \textbf{T3.3.1} Il test verifica la possibilità di eseguire il Login sulla piattaforma Facebook
\item \textbf{T3.3.2} Il test verifica il mantenimento della sessione successivo alla chiusura dell'applicazione
\item \textbf{T3.3.3} Il test verifica l'integrità dei dati allegati alla foto, inseriti dall'utente
\item \textbf{T3.3.4} Il test verifica l'esito della condivisione
\end{itemize}
\end{itemize}
\end{itemize}

 
%\chapter{Esiti Attività} % Main chapter title

\label{Capitolo6} % For referencing the chapter elsewhere, use \ref{Chapter1} 

\lhead{Capitolo 6. \emph{Esiti Attività}} % This is for the header on each page - perhaps a shortened title



\section{Esiti Analisi Statica}

Di seguito vengono riportati i risultati dell'analisi statica sul codice sorgente effettuata
con lo strumento JTest. Vengono riportati per ogni
classe:
\begin{itemize}


\item Classe: il nome della classe preceduto dall'indicazione del componente padre
\item CC: per la metrica di complessità ciclomatica viene riportato il valore ottenuto
dalla media aritmetica dei vari metodi di classe 
\item LOC: per la metrica di linee di codice viene riportato il valore intero ottenuto dalla
media aritmetica dei vari metodi di classe (per evitare di riportare i singoli
metodi)
\item LCOM HS:
il valore ottenuto dal calcolo della metrica LCOM HS che indica la
coesione dei metodi di una classe.

\end{itemize}

Il numero di campi dati e di metodi utilizzati per classe è una diretta fonte dei risultati ottenuti.


\begin{center}
    \begin{longtable}{ | p{6cm} | p{1.5cm} | p{1.5cm} | p{1.8cm} |}
    \hline
    Classe & CC & LOC & LCOM HS \\ \hline
    Activity::SplashScreenActivity& 1.2 & 5 & 0.15\\ \hline 
    Activity::FaceDetectionActivity& 3.33 & 15& 1.37\\ \hline 
    Activity::StaticFaceDetectionActivity& 1.81 & 12 & 0.8\\ \hline 
    Activity::PhotoActivity& 2.34 & 8 & 0.73\\ \hline 
    Activity::TutorialActivity& 2.55 & 4  & 0.42\\ \hline 
    Service::BootStartUpReceiver & 1 & 3 & 0.25\\ \hline 
    Service::BackgroundService & 1.33 & 5 & 0.25\\ \hline 
    Model::Data& 1.66 &  6  & 1.84\\ \hline 
    Model::Templates& 3.11 & 11  & 0\\ \hline 
    Model::Sprite& 1.82 & 5  & 0.51\\ \hline 
    Core::Camera& 1.2 & 12 & 0\\ \hline 
    Social::LoginSession& 2.73 & 5 & 0.15\\ \hline 
    Social::Share& 1.55 & 7 & 0\\ \hline 
    Utility::MatUtils& 1.27 & 5 & 0\\ \hline 
	Utility::Generics& 3.83 & 13 & 0\\ \hline 
	Adapter::CustomAdapter& 1.17 & 6 & 0.15\\ \hline     
    


    \end{longtable}
\end{center}

Complice la ridotta esperienza, durante lo sviluppo è stato necessario rivisitare ed affinare alcuni metodi per rientrare nel range di accettazione per la complessità ciclomatica scomponendo il problema in sotto-problemi di minore dimensioni e quindi più facilmente verificabili. A seguito di alcune controlli del codice tutti i limiti imposti sono stati rispettati.
Nella tabella a seguire sono presenti valori di \textit{LCOM HS} $=$ 0. Ciò possibile in quanto nella classe non sono presenti campi dati. Sono però presenti anche valori \textit{LCOM HS} > 1.0, che potrebbero indicare un campanello d'allarme. Nonostante sia difficile evitare tale problema, una migliore progettazione avrebbe potuto produrre valori di coesione migliori.


\section{Esiti Test di Validazione}

Vengono riportate le tabelle con gli esiti dei test di validazione. L'esito è stato diviso nelle seguenti categorie:
\begin{itemize}
\item \textbf{Soddisfatto:} Indica che il requisito è stato implementato con successo e quindi
soddisfatto. Colore associato \textcolor{green}{verde}
\item \textbf{Non soddisfatto:} Indica che il requisito non è stato implementato con successo.
Colore associato \textcolor{red}{rosso}
\end{itemize}

\begin{center}
    \begin{longtable}{ | p{2cm} | p{6cm} | p{2cm} | p{2cm} |}
    \hline
    Codice & Descrizione & Requisito & Esito \\ \hline
        & Verificato dal buon esito dei figli &RFO1 & \textcolor{green}{Positivo}\\ \hline 
    T1.1& Verificare il corretto funzionamento dell'attività e il caricamento dinamico dei parametri d'utilizzo in base al dispositivo utilizzato &RFO1.1 & \textcolor{green}{Positivo}\\ \hline 
    T1.2& Verificato dal buon esito dei figli  &RFO1.2 & \textcolor{green}{Positivo}\\ \hline 
    T1.2.1& Verificare la possibilità di cambiare camera a tempo d'esecuzione &RFO1.2.1 & \textcolor{green}{Positivo}\\ \hline 
    T1.2.2& Verifica la possibilità di cambiare l'elaborazione della camera (Nativa o Java) &RFO1.2.2 & \textcolor{green}{Positivo}\\ \hline 
   T1.2.3 &Verifica la possibilità di cambiare la risoluzione del frame in ingresso&RFO1.2.3 & \textcolor{red}{Non soddisfatto}\\ \hline 
    T1.2.4& Il test verifica la possibilità di cambiare il range operativo di campionamento &RFO1.2.4 & \textcolor{green}{Positivo}\\ \hline 
    T1.2.5&Il test verifica la possibilità di cambiare i parametri operativi di campiona-
mento &RFO1.2.5 & \textcolor{green}{Positivo}\\ \hline 
    T1.2.6&Il test verifica il corretto funzionamento di modifica delle parti da ricercare a
run-time &RFO1.2.6 & \textcolor{green}{Positivo}\\ \hline 
    T1.2.7&Il test verifica la possibilità di abilitare o disabilitare la modalità alto contrasto
e/o monocromatica &RFO1.2.7 & \textcolor{green}{Positivo}\\ \hline 
    T1.4& Verificato dal buon esito dei figli &RFD1.4 & \textcolor{green}{Positivo}\\ \hline 
    T1.4.1&Il test deve verificare il corretto caricamento degli Sprites &RFO1.4.1 & \textcolor{green}{Positivo}\\ \hline 
    T1.4.2&Il test deve verificare la corretta renderizzazione della ListView &RFO1.4.2 & \textcolor{red}{Non soddisfatto}\\ \hline 
    T1.4.3&Il test verifica verifica il corretto utilizzo dello Sprite assegnato &RFO1.4.3& \textcolor{green}{Positivo}\\ \hline 
    T2.1&Il test verifica il corretto caricamento di un'immagine dallo smartphone &RFO2.1  & \textcolor{green}{Positivo}\\ \hline 
    T3& Verificato dal buon esito dei figli &RFO3 & \textcolor{green}{Positivo}\\ \hline 
    T3.1& Il test verifica la possibilità di ridimensionare la foto scattata &RFD3.1 & \textcolor{green}{Positivo}\\ \hline 
    T3.2& Verificato dal buon esito dei figli &RFD3.2  & \textcolor{green}{Positivo}\\ \hline 
    T3.2.1&Il test verifica il caricamento della lista di applicazioni attraverso le quali
è possibile condividere il contenuto &RFD3.2.1  & \textcolor{red}{Non soddisfatto}\\ \hline 
    T3.2.2&Il test verifica l'esito della condivisione&RFD3.2.2  & \textcolor{green}{Positivo}\\ \hline 
    T3.3& Verificato dal buon esito dei figli &RFD3.3  & \textcolor{green}{Positivo}\\ \hline 
    T3.3.1& Il test verifica la possibilità di eseguire il Login sulla piattaforma Facebook &RFD3.3.1  & \textcolor{green}{Positivo}\\ \hline 
    T3.3.2&Il test verifica il mantenimento della sessione successivo alla chiusura
dell'applicazione &RFD3.3.1  & \textcolor{green}{Positivo}\\ \hline 
    T3.3.3& Il test verifica l'integrità dei dati allegati alla foto, inseriti dall'utente &RFD3.3.2  & \textcolor{green}{Positivo}\\ \hline 
    T3.3.4& Il test verifica il caricamento e l'esito della condivisione&RFD3.3.3  & \textcolor{green}{Positivo}\\ \hline 
    \end{longtable}
\end{center}


\section{Analisi Prestazionale}

Al fine di verificare le prestazioni ottenute in media dall'applicativo, l'azienda ha fornito 4 dispositivi che rappresentano all'anno della stesura di questo documento (2014) le possibili fasce di mercato.

I dispositivi forniti sono:

\begin{itemize}
\item \textbf{Fascia Alta - \textit{Galaxy S4}} 
	\begin{itemize}
		\item CPU quad core 1.6GHz Cortex A15 
		\item RAM 2 GB 
		\item Fotocamera : posteriore: 13 MP , anteriore 2.1 MP
		\item Android 4.2.2 Jelly Bean 
	\end{itemize}
\item \textbf{Fascia Media - \textit{Galaxy S3 Mini}}
	\begin{itemize}
		\item CPU 1 GHz dual core ARM Cortex A9
		\item RAM 1 GB 
		\item Fotocamera : posteriore: 5 MP , anteriore 1.3 MP
		\item Android 4.1.1 Jelly Bean 
	\end{itemize}
\item \textbf{Fascia Bassa - \textit{Galaxy Fame}}
	\begin{itemize}
		\item CPU 0.85 GHz single core ARM Cortex A9
		\item RAM 512 MB
		\item Fotocamera : posteriore: 3 MP , anteriore 1.3 MP
		\item Android 2.3.0 Gingerbread
	\end{itemize}
\end{itemize}

Dal momento che le prestazioni possono dipendere dalla qualità della luce e dalla distanza dell utente dal dispositivo, si è preferito testare l'applicazione utilizzando video come input. Le variabili tenute in considerazione sono risoluzione, numero di volti, numero di falsi positivi, numero di fotogrammi medi per secondo al termine di ogni video. I video sono sono volutamente caratterizzati da camere mobili, e da scene in cui non sono presenti persone al fine di verificare la presenza di falsi positivi.

I video utilizzati sono reperibili direttamente dal web ai link:

\begin{itemize}
\item[•] \href{http://www.youtube.com/watch?v=XFTbN10f_Fg}{Video 1 : Matrix - The red Woman}
\item[•] \href{http://www.youtube.com/watch?v=yWPyRSURYFQ}{Video 2 : Blade Runner - She's a Replicant}
\item[•] \href{http://www.youtube.com/watch?v=s3rv0BdxWfM}{Video 3 : The Heat - The Sun Rises and Sets With Her }
\end{itemize}

Risoluzioni utilizzate:

\begin{itemize}
\item[•] $1280\ast720$ 
\item[•] $640\ast480$ 
\item[•] $320\ast240$ 
\end{itemize}

I test sono stati effettuati con i seguenti parametri precedentemente definiti in sezione:
\begin{itemize}
\item \textit{FaceDetectionActivity::minNeighbors} = 3 (sono necessari almeno 3 hit per frame per confermare positivamente il volto)
\item \textit{FaceDetectionActivity::minSize} = -1  
\item \textit{FaceDetectionActivity::maxSize} = -1
\item \textit{FaceDetectionActivity::scaleFactor} = 1.1 
\end{itemize}

In questo modo ci si assicura che il costo computazionale per ogni frame sia massimo (il metodo di rilevamento facciale terminerà solo dopo aver valutato l'immagine nella sua interezza). 
Dato che per ogni singolo frame vi possono essere più hits corrispondenti allo stesso volto, è stata creata una semplice funzione ad hoc che permetta di contare un solo volto per set coincidenti di hits. Inoltre tali video sono stati velocizzati aumentando il framerate (24$\longmapsto$ 72 fotogrammi per secondo) per evitare qualsiasi problema di Input bound.
\\

\begin{figure}[H]\centering  
\includegraphics[scale=0.75]{/workspace/1911_up/Dropbox/thesis/Figures/fps.png}
\caption[Grafico prestazionale riassuntivo, Input Video]{Grafico prestazionale riassuntivo, Input Video}
\label{pic-a}
\end{figure}

Come previsto, al variare dei dispositivi non vi è alcuna differenza nei risultati (quantità di volti rilevati + $\varepsilon$(falsi positivi rilevati)) se non al variare del parametro \textit{FaceDetectionActivity::MinNeighbors}. Dal punto di vista puramente prestazionale, i risultati sono però stati totalmente differenti nell'utilizzo reale utilizzando la fotocamera posteriore come input (contrariamente ai video che garantivano un framerate costante di 72 fotogrammi per secondo). 


\begin{figure}[H]\centering  
\includegraphics[scale=0.75]{/workspace/1911_up/Dropbox/thesis/Figures/luce.png}
\caption[Grafico prestazionale riassuntivo, Condizioni Ottimali, Input Camera]{Grafico prestazionale riassuntivo, Condizioni Ottimali, Input Camera}
\label{pic-a}
\end{figure}




\begin{figure}[H]\centering  
\includegraphics[scale=0.75]{/workspace/1911_up/Dropbox/thesis/Figures/buio.png}
\caption[Grafico prestazionale riassuntivo, Bassa luminosità, Input Camera]{Grafico prestazionale riassuntivo, Bassa luminosità, Input Camera}
\label{pic-a}
\end{figure}



Infine si può notare come vi siano grosse discrepanze di prestazioni al variare della fotocamera d'input (Anteriore o Posteriore). Questo è dovuto al fatto che generalmente la fotocamera posteriore è di qualità maggiore (sensore di dimensioni maggiori, diaframma più aperto ed ISO che può raggiungere valori più elevati)
\\
\begin{figure}[H]\centering  
\includegraphics[scale=0.75]{/workspace/1911_up/Dropbox/thesis/Figures/antvspost.png}
\caption[Grafico prestazionale, differenza tra camera anteriore e posteriore]{Grafico prestazionale, differenza tra camera anteriore e posteriore}
\label{pic-a}
\end{figure}

\subsection{Considerazioni}

A seguito dei test effettuati l'idea è che i dispositivi correntemente presenti sul mercato abbiamo una potenza computazionale più che sufficiente per eseguire tali task con prestazioni discrete e garantiscano un'esperienza d'utilizzo adeguata. Il maggior problema riscontrato è sicuramente la qualità della fotocamera che può condizionare drasticamente le performance se posta in condizione d'utilizzo non ottimale. Il software preinstallato sullo smartphone infatti, in condizioni di bassa luminosità, tende ad alzare il valore ISO (sensibilità del sensore) ed ad utilizzare un diaframma il più aperto possibile. La necessità di miniaturizzare l'hardware fotografico ha però come conseguenza quella di non poter avere sensori e lenti di dimensioni adeguate. Questo si traduce in bassa qualità, e pertanto per ogni singolo frame di un video sarà necessario un tempo di esposizione molto elevato. Lo stream di fotogrammi risulterà quindi caratterizzato da un framerate molto basso, ponendo il processore in attesa attiva del fotogramma da processare. 

 
%% Chapter 1

\chapter{Valutazione retrospettiva} % Main chapter title

\label{Chapter4} % For referencing the chapter elsewhere, use \ref{Chapter1} 

\lhead{Capitolo 4. \emph{Valutazione Retrospettiva}} % This is for the header on each page - perhaps a shortened title

%----------------------------------------------------------------------------------------

\section{Copertura dei Requisiti}

Vengono riportate le tabelle con gli esiti dell'attività di sviluppo. Per ogni
requisito viene riportato l'esito dello stesso dopo lo sviluppo e l'implementazione del
software. L'esito è stato diviso nelle seguenti categorie:
\begin{itemize}
\item \textbf{Soddisfatto:} Indica che il requisito è stato implementato con successo e quindi
soddisfatto. Colore associato \textcolor{green}{verde}
\item \textbf{Non soddisfatto:} Indica che il requisito non è stato implementato con successo.
Colore associato \textcolor{red}{rosso}
\item \textbf{Incompleto:} Indica che il requisito è stato implementato solo in alcune sue
parti o che il risultato non è stato raggiunto in modo soddisfacente. Colore associato \textcolor{gray}{grigio}
\end{itemize}

\begin{center}
    \begin{longtable}{ | p{2cm} | p{7cm} | p{2cm} |}
    \hline
    Requisiti & Descrizione & Esito \\ \hline
    RFO1 &  L'applicazione deve permettere il rilevamento facciale in tempo reale su dispositivi mobili Android & \textcolor{green}{Soddisfatto}  \\ \hline 
    RFO1.1 &  L'applicazione deve permettere di avviare il training OpenCV  & \textcolor{green}{Soddisfatto} \\ \hline
    RFO1.2 &  L'applicazione deve permettere di calibrare la camera corrente  & \textcolor{green}{Soddisfatto}  \\ \hline 
    RFO1.2.1 &  L'applicazione deve permettere all'utente di selezionare la camera desiderata & \textcolor{green}{Soddisfatto} \\ \hline
    RFO1.2.2 &  L'applicazione deve permettere all'utente di modificare il tipo di elaborazione della camera & \textcolor{green}{Soddisfatto} \\ \hline
    RFO1.2.3 &  L'applicazione deve permettere all'utente di modificare la risoluzione del frame in ingresso & \textcolor{green}{Soddisfatto}  \\ \hline
    RFO1.2.4 &  L'applicazione deve permettere all'utente di modificare la profondità di campionamento (distanza minima e massima su cui effettuare lo scan) & \textcolor{green}{Soddisfatto}  \\ \hline
    RFO1.2.5 &  L'applicazione deve permettere all'utente di modificare il numero minimo di campionamenti sotto il quale il volto non viene riconosciuto come valido  & \textcolor{green}{Soddisfatto}  \\ \hline
    RFO1.2.6 &  L'applicazione deve permettere all'utente di abilitare di abilitare/disabilitare la ricerca delle features in tempo reale  & \textcolor{green}{Soddisfatto}  \\ \hline
    RFO1.2.7 &  L'applicazione deve permettere all'utente di abilitare/disabilitare la modalità contrasto e/o monocromatica  & \textcolor{green}{Soddisfatto}  \\ \hline
    RFD1.3 &  L'applicazione deve permettere di cogliere l'orientamento dell'eventuale volto individuato & \textcolor{gray}{Incompleto} \\ \hline    
    RFD1.4 &  L'applicazione deve permettere di scegliere lo Sprite da utilizzare & \textcolor{green}{Soddisfatto} \\ \hline    
    RFO1.4.1 &  L'applicazione deve caricare una lista di Sprites, che saranno successivamente selezionabili dall'utente & \textcolor{green}{Soddisfatto}  \\ \hline
    RFO1.4.2 &  L'applicazione deve permettere all'utente di selezionare uno o più Sprites desiderati da una ListView presente all'interno dell'interfaccia grafica  & \textcolor{green}{Soddisfatto}  \\ \hline
    RFO1.4.3 &  L'applicazione deve poter renderizzare gli Sprites precedentemente selezionati & \textcolor{green}{Soddisfatto} \\ \hline    
    RFO2 &  L'applicazione deve permettere di rilevare volti a partire da un immagine statica su dispositivi mobili Android & \textcolor{green}{Soddisfatto}  \\ \hline
    RFO2.1 &  L'applicazione deve permettere all'utente di caricare una foto dal proprio smartphone & \textcolor{green}{Soddisfatto}  \\ \hline
    RFO2.2 &  L'applicazione deve permettere di avviare il rilevamento facciale & \textcolor{green}{Soddisfatto} \\ \hline
    RFD2.3 &   L'applicazione deve permettere di scegliere uno o più Sprites da utilizzare  & \textcolor{green}{Soddisfatto}  \\ \hline 
    RFO3 &  L'applicazione deve permettere la condivisione di contenuti multimediali & \textcolor{green}{Soddisfatto} \\ \hline
    RFD3.1 &  L'applicazione deve permettere di ridimensionare la foto scattata in base a formati predefiniti & \textcolor{green}{Soddisfatto} \\ \hline
    RFD3.2 &  L'applicazione deve permettere di condividere la foto ottenuta verso applicazioni di messaggistica interne  & \textcolor{green}{Soddisfatto} \\ \hline
    RFD3.2.1 &  L'applicazione deve fornire l'utente di un bottone attraverso il quale può esprimere il suo intento di condividere il contenuto  & \textcolor{green}{Soddisfatto} \\ \hline
    RFD3.2.2 &  L'applicazione deve rispondere fornendo graficamente una risposta contenente i servizi di inoltro disponibili  & \textcolor{green}{Soddisfatto} \\ \hline
    RFD3.2.3 &  L'applicazione deve condividere il contenuto multimediale attraverso il servizio selezionato  & \textcolor{green}{Soddisfatto} \\ \hline    
    RFD3.3 &  L'applicazione deve permettere il login facebook e condividere la foto su quest ultimo  & \textcolor{green}{Soddisfatto} \\ \hline
    RFD3.3.1 &  L'applicazione deve permettere l'inoltro di una richiesta di accesso verso i server Facebook & \textcolor{green}{Soddisfatto} \\ \hline
    RFD3.3.2 &  L'applicazione deve permettere l'inserimento di un testo da allegare alla foto & \textcolor{green}{Soddisfatto} \\ \hline
    RFD3.3.3 &  L'applicazione deve permettere il caricamento della foto sulla piattaforma Facebook & \textcolor{green}{Soddisfatto} \\ \hline
    RFD4 &  L'applicazione deve mantenere un framerate minimo di 15 fotogrammi al secondo & \textcolor{red}{Non Soddisfatto} \\ \hline
    \end{longtable}
\end{center}


%----------------------------------------------------------------------------------------
\newpage
\section{Prodotto finale}



\section{Conoscenze acquisite}

Durante l'esperienza di stage ho approfondito le mie capacità in ognuna delle fasi principali
dello sviluppo software. Ho avuto modo di sviluppare competenze in diverse tecnologie e nel loro uso. Le più rilevanti dal mio punto di vista sono le seguenti:

\begin{itemize}
\item \textbf{Java} Le precedenti conoscenze teoriche apprese durante il corso di Programmazione Concorrente e Distribuita sono state applicate in maniera concreta di fronte a problemi reali. 
\item \textbf{Android Studio} Le conoscenze riguardo a questo ambiente prima dell'inizio dello stage erano pressochè nulle in quanto non mi ero mai trovato ad affrontare
questo argomento. Ho avuto modo di conoscere ed ad utilizzare parte dell'enorme mole di strumenti offerti allo sviluppatore, in particolar modo quelli dediti al debug (DDMS), all'analisi della memoria e alla suite di test.
\item \textbf{Facebook} Le conoscenze riguardo a questo ambiente prima dell'inizio dello stage erano pressochè nulle. Grazie ai numerosi tutorial ufficiali presenti in rete è stato possibile realizzare facilmente i requisiti definiti.
\item \textbf{OpenCV} Lo studio di tale libreria mi ha permesso di conoscere, nei limiti del tempo concesso per lo stage, questo variegato strumento di analisi ed elaborazione delle immagini. In particolar modo è stato molto interessante lo studio di come gli oggetti vengono campionati per poi essere utilizzati all'interno di un classificatore. Inoltre, a fini prestazionali, è stata significativa l'analisi effettuata per variare i parametri di rilevamento in base all'hardware presente sul dispositivo. 
\end{itemize}

%----------------------------------------------------------------------------------------



%----------------------------------------------------------------------------------------

\section{Conoscenze possedute e relazione con la preparazione accademica}

Durante lo svolgimento dello stage si sono potute applicare le conoscenze già
acquisite per la gestione di progetto e per il miglior sviluppo possibile dell'applicazione,
potendo provare sul campo l'importanza dell'analisi dei requisiti e della
validazione, i benefici di una attenta progettazione dell'architettura e
di una strategia formale ed estensiva di test del prodotto.
Credo che la preparazione accademica precedente all'attività di stage
sia stata adeguata perché potessi affrontare al meglio il progetto, non tanto
nelle conoscenze fornite quanto piuttosto nel metodo insegnato durante tutti i
corsi del percorso di laurea triennale. Inoltre il supporto del tutor aziendale Marcus Oblack mi ha dato la motivazione e la curiosità di informarmi sulle tecnologie necessarie per condurre in maniera efficace la creazione di questo progetto.

Per concludere penso che questo percorso sia stato impegnativo, vivo e stimolante. Mi ha
permesso di mettere in pratica le conoscenze e le capacità acquisite durante la mia vita ed il mio il percorso di studi. Mi ritengo quindi pienamente soddisfatto del suo esito finale.

 

%----------------------------------------------------------------------------------------
%	THESIS CONTENT - APPENDICES
%----------------------------------------------------------------------------------------

\addtocontents{toc}{\vspace{2em}} % Add a gap in the Contents, for aesthetics

\appendix % Cue to tell LaTeX that the following 'chapters' are Appendices

% Include the appendices of the thesis as separate files from the Appendices folder
% Uncomment the lines as you write the Appendices

% Appendix A

\newpage
\chapter{Glossario} % Main appendix title
\renewcommand{\chaptername}{MyChapter}
\label{AppendixA} % For referencing this appendix elsewhere, use \ref{AppendixA}

\lhead{\emph{Glossario}} % This is for the header on each page - perhaps a shortened title

\textbf{Ant:} libreria Java il cui compito è quello di guidare il processo di build di applicazioni Java. Fornisce numerose funzionalità in grado di di compilare, assemblare, testare ed avviare le applicazioni.


\textbf{Diaframma:} In fotografia ed in ottica, un diaframma è un'apertura solitamente circolare o poligonale, incorporata nel barilotto dell'obiettivo, che ha il compito di controllare la quantità di luce che raggiunge la pellicola (in una macchina fotografica analogica) o i sensori (in una macchina fotografica digitale) nel tempo in cui l'otturatore resta aperto (tempo di esposizione).

\textbf{Gradle:} recente strumento di build che Google vuole portare a standard per Android. Caratteristiche ne sono la maggiore estensibilità ed utilizzabilità rispetto ad Ant.


% Appendix A

\chapter{Riferimenti} % Main appendix title

\label{Riferimenti} % For referencing this appendix elsewhere, use \ref{AppendixA}

\lhead{\emph{Riferimenti}} % This is for the header on each page - perhaps a shortened title


%\input{Appendices/AppendixC}

\addtocontents{toc}{\vspace{2em}} % Add a gap in the Contents, for aesthetics

\backmatter

%----------------------------------------------------------------------------------------
%	BIBLIOGRAPHY
%----------------------------------------------------------------------------------------

%\label{Bibliography}

%\lhead{\emph{Bibliography}} % Change the page header to say "Bibliography"

%\bibliographystyle{unsrtnat} % Use the "unsrtnat" BibTeX style for formatting the Bibliography

%\bibliography{Bibliography} % The references (bibliography) information are stored in the file named "Bibliography.bib"

\end{document}