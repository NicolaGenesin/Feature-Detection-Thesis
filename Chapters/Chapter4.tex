% Chapter 1

\chapter{Valutazione retrospettiva} % Main chapter title

\label{Chapter4} % For referencing the chapter elsewhere, use \ref{Chapter1} 

\lhead{Capitolo 4. \emph{Valutazione Retrospettiva}} % This is for the header on each page - perhaps a shortened title

%----------------------------------------------------------------------------------------

\section{Copertura dei Requisiti}

Vengono riportate le tabelle con gli esiti dell'attività di sviluppo. E' presente una
tabella per ogni tipologia di requisito (di qualità, funzionale, di vincolo). Per ogni
requisito viene riportato l'esito dello stesso dopo lo sviluppo e l'implementazione del
software. L'esito è stato diviso nelle seguenti categorie:
\begin{itemize}
\item \textbf{Soddisfatto:} Indica che il requisito è stato implementato con successo e quindi
soddisfatto. Colore associato \textcolor{green}{verde}
\item \textbf{Non soddisfatto:} Indica che il requisito non è stato implementato con successo.
Colore associato \textcolor{red}{rosso}
\item \textbf{Incompleto:} Indica che il requisito è stato implementato solo in alcune sue
parti. Colore associato \textcolor{yellow}{giallo}
\end{itemize}

\begin{center}
    \begin{tabular}{ | l | l | l | p{5cm} |}
    \hline
    Codice & Descrizione & Tipologia & Esito \\ \hline
    test & test & test & \textcolor{yellow}{giallo} \\ \hline
    test & test & test & \textcolor{yellow}{giallo} \\ \hline
    test & test & test & \textcolor{yellow}{giallo} \\ \hline
    test & test & test & \textcolor{yellow}{giallo} \\ \hline
    test & test & test & \textcolor{yellow}{giallo} \\ \hline
    test & test & test & \textcolor{yellow}{giallo} \\ \hline
    test & test & test & \textcolor{yellow}{giallo} \\ \hline
    test & test & test & \textcolor{yellow}{giallo} \\ \hline
    

    \hline
    \end{tabular}
\end{center}


%----------------------------------------------------------------------------------------

\section{Conoscenze acquisite}

%----------------------------------------------------------------------------------------

\section{Distanza tra conoscenze richieste e conoscenze possedute}

%----------------------------------------------------------------------------------------

