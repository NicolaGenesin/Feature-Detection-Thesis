\chapter{Verifica e Validazione} % Main chapter title

\label{Capitolo5} % For referencing the chapter elsewhere, use \ref{Chapter1} 

\lhead{Capitolo 5. \emph{Verifica e Validazione}} % This is for the header on each page - perhaps a shortened title

\section{Analisi Statica}
Al fine di portare avanti il processo di verifica con metodo è stato necessario renderlo
quantificabile. Perciò si sono definite delle metriche sul codice sorgente. Esse sono
di seguito definite e il loro significato viene descritto e spiegato. Per ogni metrica si
sono definiti un range di accettazione e un range ottimale.

\subsection{Complessità ciclomatica}

Tale metrica indica il numero di cammini linearmente indipendenti percorribili durante
l'esecuzione di un singolo metodo. Tale metrica è molto importante in quanto ha
implicazioni dirette sulle attività di testing: infatti essa rappresenta un upper bound
per il numero di routine di test necessarie per raggiungere un completo branch coverage.

\textbf{Range dichiarato:}
\begin{itemize}
\item Accettazione $\leq 15$
\item Ottimale $\leq 10$
\end{itemize}

\subsection{LOC}

Tale metrica (Lines of Code, linee di codice) indica il numero di linee di codice. Nel conteggio vengono escluse le righe contenenti dichiarazioni di namespaces, tipi, campi, metodi oltre che i metodi astratti e i tipi enumeration.
Solo il codice effettivamente eseguito è considerato nel conteggio delle righe.
La metrica LOC non è sempre legata alla produttività del programmatore, ma torna
utile sia nel calcolo della percentuale di statement coverage e nella valutazione del
software.
Essa è indice di manutenibilià (è più semplice manutenere metodi brevi) e quindi
qualità del codice.

\textbf{Range per metodo:}
\begin{itemize}
\item Accettazione $\leq 45$
\item Ottimale $\leq 15$
\end{itemize}

\subsection{Numero di campi utilizzati per classe}

Indica il numero di membri di classe (campi dati) di una particolare classe. E'
importante in quanto è indice delle responsabilità andate ad una classe, se viene
superato può indicare che la classe ne raggruppa troppe altre e andrebbe ulteriormente
separata in classi distinte.

\textbf{Range dichiarato:}
\begin{itemize}
\item Accettazione $\leq 16$
\item Ottimale $\leq 8$
\end{itemize}

\subsection{Numero di metodi utilizzati per classe}

Indica il numero di metodi di una particolare classe. Similmente al numero di campi
per classe è importante in quanto è indice delle responsabilità andate ad una classe, se
viene superato può indicare che la classe ha troppe funzionalità e andrebbe separata.

\textbf{Range dichiarato:}
\begin{itemize}
\item Accettazione $\leq 16$
\item Ottimale $\leq 8$
\end{itemize}

\subsection{LCOM e LCOM HS}

Si vuole definire una metrica a partire da questa definizione: se una classe ha una sola responsabilità si
dice che essa è coesa. In generale una classe risulterà coesa se i suoi metodi sono
strettamente legati fra loro. Quando metodi distinti non usano attributi o metodi
comuni significa che essi non condividono nulla e che quindi potrebbero venir separati.
La metrica di LCOM misura quindi quanto poco una classe è coesa. Alcune formule per il
calcolo di LCOM sono le seguenti:

\begin{center}
\textbf{LCOM} =  $1-(\sum(MF)/M \ast F) $  \\
\textbf{LCOM HS} = $(M-(\sum(MF)/F))\ast(M-1)$
\end{center}

Dove:
\begin{itemize}
\item M è il numero di metodi nella classe. Si considerano sia i metodi statici sia quelli di istanza, ed include inoltre i costruttori,getters e setters ed eventuali metodi del tipo add/remove
\item F è il numero di campi dati d'istanza all'interno della classe 
\item MF è il numero di metodi della classe che accedono un particolare campo dati di tipo classe
\item $\sum(MF)$ è la somma degli MF tra tutti i campi d'istanza della classe
\end{itemize}

L'idea alla base di tali formule è che una classe perfettamente coesa utilizza tutti i
suoi campi dati di tipo classe all'interno di ogni suo metodo il che comporta :

\begin{center}
 $(\sum(MF) = M \ast F) \Leftrightarrow LCOM = 0 $
\end{center}

La differenza tra LCOM e la sua versione HS (Hendersons-Sellers) è che la prima
ritorna dei valori nel range [0-1] mentre la seconda nel range [1-2]. La versione HS è
considerata più efficace per la rilevazione dei tipi scarsamente coesi.
Tale metrica è interessante, ma va trattata con cautela (basti pensare una classe
che ha n campi dati, n getters e n setters, essa risulterà scarsamente coesa, il che
non rispecchia la realtà). Infatti essa non va valutata singolarmente, ma deve essere
inserita in un contesto di valutazione che comprende altre variabili, soprattutto il
numero di campi e il numero di metodi. 

\textbf{Range dichiarato:}
\begin{itemize}
\item Accettazione $\leq 0.8$
\item Ottimale = 0
\end{itemize}

\newpage
\section{Analisi Dinamica}

Nello sviluppo dell'applicazione si sono progettati dei test che precedessero l'implementazione di ogni componente e di ogni classe. I test di integrazione hanno definito come i vari
componenti avrebbero dovuto relazionarsi tra di loro, e per ogni componente sono stati definiti adeguati test di unità che andassero a verificare il corretto funzionamento delle classi interne. Ogni unità e stata testata in
relazione al ruolo che essa avrebbe avuto all'interno del sistema in base al suo "peso".
I test d'unità verificano in gran parte l'integrità degli input inseriti dall'utente e di come questi variano durante l'utilizzo dell'applicativo.

Per l'analisi delle chiamate ai metodi forniti dalle librerie esterne (OpenCV e Facebook SDK) si è cercato di
verificare la correttezza nei limiti di risorse e tempo disponibili. Infatti alcune funzioni fornite da tali librerie sono ad alto livello e comprendono un grafo di chiamate molto esteso
(in particolare chiamate a metodi di OpenCV, come il rilevamento facciale, e a Facebook per quanto riguarda il login e il mantenimento della sessione attiva), e non è stato possibile progettare test in quanto tali operazioni sono
particolarmente complesse e avrebbero richiesto una profonda comprensione della meccanica di tali librerie.

Il tool utilizzato per eseguire i test è stato Android JUnit che fornisce delle classi previste di metodi adibiti alla creazione dei cosiddetti oggetti mock (oggetti simili a quelli reali ma che hanno l'obiettivo di simulare l'oggetto reale in maniera controllata) e metodi che possano aiutare ad aiutare il ciclo di vita di un componente.

\section{Test di Validazione}

Vengono riportati in questa sezione i test che descrivono la strategia utilizzata per la
validazione del prodotto. Si è cercato di tenere un rapporto di 1:1 tra i
requisiti funzionali e i test di validazione che verificavano quella specifica funzionalità.
Per ogni test viene riportato il suo codice gerarchico identificativo. Se tale test risulta sufficientemente grande, esso viene scomposto in sotto-test aventi come indice gerarchico l'indice del padre.

\begin{itemize}
\item \textbf{T1} Il test è atto a verificare il buon funzionamento dell'attività di riconoscimento facciale e delle rispettive parti
\begin{itemize}
\item \textbf{T1.1} Il test verifica il corretto caricamento delle librerie e le adatta in base all'hardware caratterizzante dispositivo
\item \textbf{T1.2} Il test verifica la possibilità di eseguire la calibrazione con le
impostazioni selezionate
\begin{itemize}
\item \textbf{T1.2.1} Il test verifica la possibilità di cambiare camera a run-time
\item \textbf{T1.2.2} Il test verifica la possibilità di cambiare il tipo di elaboratore della camera
\item \textbf{T1.2.3} Il test verifica la possibilità di cambiare la risoluzione del frame in ingresso
\item \textbf{T1.2.4} Il test verifica la possibilità di cambiare il range operativo di campionamento
\item \textbf{T1.2.5} Il test verifica la possibilità di cambiare i parametri operativi di campionamento
\item \textbf{T1.2.6} Il test verifica il corretto funzionamento di modifica delle parti da ricercare a run-time
\item \textbf{T1.2.7} Il test verifica la possibilità di abilitare o disabilitare la modalità alto contrasto e/o monocromatica
\end{itemize}


\item \textbf{T1.4} Il test verifica la possibilità di cambiare lo Sprite corrente
\begin{itemize}
\item \textbf{T1.4.1} Il test deve verificare il corretto caricamento degli Sprites
\item \textbf{T1.4.2} Il test deve verificare la corretta renderizzazione della ListView
\item \textbf{T1.4.3} Il test verifica verifica il corretto utilizzo dello Sprite assegnato
\end{itemize}
\end{itemize}
\item \textbf{T2.1} Il test verifica il corretto caricamento di un'immagine dallo smartphone
\item \textbf{T3} Il test verifica la possibilità di condividere la foto
\begin{itemize}
\item \textbf{T3.1} Il test verifica la possibilità di ridimensionare la foto scattata
\item \textbf{T3.2} Il test verifica la possibilità di condividere la foto inoltrandola verso applicazioni di messaggistica preinstallate nello smartphone
\begin{itemize}
\item \textbf{T3.2.1} Il test verifica il caricamento della lista di applicazioni attraverso le quali è possibile condividere il contenuto
\item \textbf{T3.2.2} Il test verifica l'esito della condivisione
\end{itemize}
\item \textbf{T3.3} Il test verifica la possibilità di eseguire la connessione verso la piattaforma Facebook
\begin{itemize}
\item \textbf{T3.3.1} Il test verifica la possibilità di eseguire il Login sulla piattaforma Facebook
\item \textbf{T3.3.2} Il test verifica il mantenimento della sessione successivo alla chiusura dell'applicazione
\item \textbf{T3.3.3} Il test verifica l'integrità dei dati allegati alla foto, inseriti dall'utente
\item \textbf{T3.3.4} Il test verifica l'esito della condivisione
\end{itemize}
\end{itemize}
\end{itemize}

