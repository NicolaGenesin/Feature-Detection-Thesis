% Chapter 1

\chapter{Valutazione retrospettiva} % Main chapter title

\label{Chapter7} % For referencing the chapter elsewhere, use \ref{Chapter1} 

\lhead{Capitolo 7. \emph{Valutazione Retrospettiva}} % This is for the header on each page - perhaps a shortened title

%----------------------------------------------------------------------------------------

\section{Copertura dei Requisiti}

Vengono riportate le tabelle con gli esiti dell'attività di sviluppo. Per ogni
requisito viene riportato l'esito dello stesso dopo lo sviluppo e l'implementazione del
software. L'esito è stato diviso nelle seguenti categorie:
\begin{itemize}
\item \textbf{Soddisfatto:} Indica che il requisito è stato implementato con successo e quindi
soddisfatto. Colore associato \textcolor{green}{verde}
\item \textbf{Non soddisfatto:} Indica che il requisito non è stato implementato con successo.
Colore associato \textcolor{red}{rosso}
\item \textbf{Incompleto:} Indica che il requisito è stato implementato solo in alcune sue
parti o che il risultato non è stato raggiunto in modo soddisfacente. Colore associato \textcolor{gray}{grigio}
\end{itemize}

\begin{center}
    \begin{longtable}{ | p{2cm} | p{7cm} | p{2cm} |}
    \hline
    Requisiti & Descrizione & Esito \\ \hline
    RFO1 &  L'applicazione deve permettere il rilevamento facciale in tempo reale su dispositivi mobili Android & \textcolor{green}{Soddisfatto}  \\ \hline 
    RFO1.1 &  L'applicazione deve permettere di avviare il training OpenCV  & \textcolor{green}{Soddisfatto} \\ \hline
    RFO1.2 &  L'applicazione deve permettere di calibrare la camera corrente  & \textcolor{green}{Soddisfatto}  \\ \hline 
    RFO1.2.1 &  L'applicazione deve permettere all'utente di selezionare la camera desiderata & \textcolor{green}{Soddisfatto} \\ \hline
    RFO1.2.2 &  L'applicazione deve permettere all'utente di modificare il tipo di elaborazione della camera & \textcolor{green}{Soddisfatto} \\ \hline
    RFO1.2.3 &  L'applicazione deve permettere all'utente di modificare la risoluzione del frame in ingresso & \textcolor{green}{Soddisfatto}  \\ \hline
    RFO1.2.4 &  L'applicazione deve permettere all'utente di modificare la profondità di campionamento (distanza minima e massima su cui effettuare lo scan) & \textcolor{green}{Soddisfatto}  \\ \hline
    RFO1.2.5 &  L'applicazione deve permettere all'utente di modificare il numero minimo di campionamenti sotto il quale il volto non viene riconosciuto come valido  & \textcolor{green}{Soddisfatto}  \\ \hline
    RFO1.2.6 &  L'applicazione deve permettere all'utente di abilitare di abilitare/disabilitare la ricerca delle features in tempo reale  & \textcolor{green}{Soddisfatto}  \\ \hline
    RFO1.2.7 &  L'applicazione deve permettere all'utente di abilitare/disabilitare la modalità contrasto e/o monocromatica  & \textcolor{green}{Soddisfatto}  \\ \hline
    RFD1.3 &  L'applicazione deve permettere di cogliere l'orientamento dell'eventuale volto individuato & \textcolor{gray}{Incompleto} \\ \hline    
    RFD1.4 &  L'applicazione deve permettere di scegliere lo Sprite da utilizzare & \textcolor{green}{Soddisfatto} \\ \hline    
    RFO1.4.1 &  L'applicazione deve caricare una lista di Sprites, che saranno successivamente selezionabili dall'utente & \textcolor{green}{Soddisfatto}  \\ \hline
    RFO1.4.2 &  L'applicazione deve permettere all'utente di selezionare uno o più Sprites desiderati da una ListView presente all'interno dell'interfaccia grafica  & \textcolor{green}{Soddisfatto}  \\ \hline
    RFO1.4.3 &  L'applicazione deve poter renderizzare gli Sprites precedentemente selezionati & \textcolor{green}{Soddisfatto} \\ \hline    
    RFO2 &  L'applicazione deve permettere di rilevare volti a partire da un immagine statica su dispositivi mobili Android & \textcolor{green}{Soddisfatto}  \\ \hline
    RFO2.1 &  L'applicazione deve permettere all'utente di caricare una foto dal proprio smartphone & \textcolor{green}{Soddisfatto}  \\ \hline
    RFO2.2 &  L'applicazione deve permettere di avviare il rilevamento facciale & \textcolor{green}{Soddisfatto} \\ \hline
    RFD2.3 &   L'applicazione deve permettere di scegliere uno o più Sprites da utilizzare  & \textcolor{green}{Soddisfatto}  \\ \hline 
    RFO3 &  L'applicazione deve permettere la condivisione di contenuti multimediali & \textcolor{green}{Soddisfatto} \\ \hline
    RFD3.1 &  L'applicazione deve permettere di ridimensionare la foto scattata in base a formati predefiniti & \textcolor{green}{Soddisfatto} \\ \hline
    RFD3.2 &  L'applicazione deve permettere di condividere la foto ottenuta verso applicazioni di messaggistica interne  & \textcolor{green}{Soddisfatto} \\ \hline
    RFD3.2.1 &  L'applicazione deve fornire l'utente di un bottone attraverso il quale può esprimere il suo intento di condividere il contenuto  & \textcolor{green}{Soddisfatto} \\ \hline
    RFD3.2.2 &  L'applicazione deve rispondere fornendo graficamente una risposta contenente i servizi di inoltro disponibili  & \textcolor{green}{Soddisfatto} \\ \hline
    RFD3.2.3 &  L'applicazione deve condividere il contenuto multimediale attraverso il servizio selezionato  & \textcolor{green}{Soddisfatto} \\ \hline    
    RFD3.3 &  L'applicazione deve permettere il login facebook e condividere la foto su quest ultimo  & \textcolor{green}{Soddisfatto} \\ \hline
    RFD3.3.1 &  L'applicazione deve permettere l'inoltro di una richiesta di accesso verso i server Facebook & \textcolor{green}{Soddisfatto} \\ \hline
    RFD3.3.2 &  L'applicazione deve permettere l'inserimento di un testo da allegare alla foto & \textcolor{green}{Soddisfatto} \\ \hline
    RFD3.3.3 &  L'applicazione deve permettere il caricamento della foto sulla piattaforma Facebook & \textcolor{green}{Soddisfatto} \\ \hline
    RFD4 &  L'applicazione deve mantenere un framerate minimo di 15 fotogrammi al secondo & \textcolor{red}{Non Soddisfatto} \\ \hline
    \end{longtable}
\end{center}


%----------------------------------------------------------------------------------------
\newpage
\section{Prodotto finale}

Il progetto realizzato ha raggiunto un discreto grado di maturità. L'intento dell'Azienda di verificare la fattibilità di questo progetto ha avuto un esito che, seppur a posteriori, non è facile valutare. Nonostante non vi siano stati grandi problemi di integrazione con le librerie utilizzate, le prestazioni che ci si aspettava sono state raggiunte solo per terminali mobili di fascia alta del mercato. Lag e rallentamenti possono invece pregiudicare negativamente l'esperienza d'utilizzo dell'applicazione nei terminali di fascia media. Data l'inesperienza su tale campo, ritengo che sarebbe stato opportuno uno studio di fattibilità più approfondito per evidenziare meccaniche o tecnologie potenzialmente più performanti. Nonostante la potenza computazionale degli odierni smartphone sia in continua crescita, ritengo comunque sia necessario creare una struttura scalabile che sia il più possibile indipendente dall'hardware su cui essa dovrà girare, e che non si adagi sulla potenza bruta piuttosto che su un buon sistema. 

Personalmente ritengo che l'utilizzo delle attuali librerie wrapper Java per OpenCV non possano garantire quanto appena definito (sono infatti numerose le chiamate JNI verso codice nativo C/C++ aventi un costo molto elevato). 

Dal punto di vista funzionale, l'applicazione rispetta gli obiettivi definiti nelle fasi preliminari con le seguenti caratteristiche:

\begin{itemize}
\item accuratezza elevata per volti perfettamente frontali
\item accuratezza elevata per volti di profilo (utilizzando un classificatore differente)
\item accuratezza scarsa per volti inclinati e/o ruotati
\end{itemize}

Tali caratteristiche sono però relative in quanto dipendono da quale classificatore viene utilizzato per il rilevamento della parte indicata. Nel corso di tale progetto sono stati individuati classificatori istruiti di dimensioni molto ridotte in modo da ridurre al minimo il costo di computazione (\textit{$http://alereimondo.no-ip.org/OpenCV/34$}). Se si volesse perseguire questa strada, sarebbe opportuno creare un classificatore adatto alle specifiche esigenze dell'applicazione.

\section{Conoscenze acquisite}

Durante l'esperienza di stage ho approfondito le mie capacità in ognuna delle fasi principali
dello sviluppo software. Ho avuto modo di sviluppare competenze in diverse tecnologie e nel loro uso. Le più rilevanti dal mio punto di vista sono le seguenti:

\begin{itemize}
\item \textbf{Java} Le precedenti conoscenze teoriche apprese durante il corso di Programmazione Concorrente e Distribuita sono state applicate in maniera concreta di fronte a problemi reali. 
\item \textbf{Android Studio} Le conoscenze riguardo a questo ambiente prima dell'inizio dello stage erano pressochè nulle in quanto non mi ero mai trovato ad affrontare
questo argomento. Ho avuto modo di conoscere ed ad utilizzare parte dell'enorme mole di strumenti offerti allo sviluppatore, in particolar modo quelli dediti al debug (DDMS), all'analisi della memoria e alla suite di test.
\item \textbf{Facebook} Le conoscenze riguardo a questo ambiente prima dell'inizio dello stage erano pressochè nulle. Grazie ai numerosi tutorial ufficiali presenti in rete è stato possibile realizzare facilmente i requisiti richiesti.
\item \textbf{OpenCV} Lo studio di tale libreria mi ha permesso di conoscere, nei limiti del tempo concesso per lo stage, questo variegato strumento di analisi ed elaborazione delle immagini. In particolar modo è stato molto interessante lo studio di come gli oggetti vengono campionati per poi essere utilizzati all'interno di un classificatore. Inoltre, a fini prestazionali, è stata significativa l'analisi effettuata per variare i parametri di rilevamento in base all'hardware presente sul dispositivo. La vastità di questa tecnologia sarà sicuramente materia di interesse nei miei futuri progetti. 
\end{itemize}

%----------------------------------------------------------------------------------------



%----------------------------------------------------------------------------------------

\section{Conoscenze possedute e relazione con la preparazione accademica}

Durante lo svolgimento dello stage si sono potute applicare le conoscenze già
acquisite per la gestione di progetto e per il miglior sviluppo possibile dell'applicazione,
potendo provare sul campo l'importanza dell'analisi dei requisiti e della
validazione, i benefici di una attenta progettazione dell'architettura e
di una strategia formale ed estensiva di test del prodotto.
Credo che la preparazione accademica precedente all'attività di stage
sia stata adeguata perché potessi affrontare al meglio il progetto, non tanto
nelle conoscenze fornite quanto piuttosto nel metodo insegnato durante tutti i
corsi del percorso di laurea triennale. Inoltre il supporto del tutor aziendale Marcus Oblack mi ha dato la motivazione e la curiosità di informarmi sulle tecnologie necessarie per condurre in maniera efficace la creazione di questo progetto.

Per concludere penso che questo percorso sia stato impegnativo, vivo e stimolante. Mi ha
permesso di mettere in pratica le conoscenze e le capacità acquisite durante la mia vita ed il mio il percorso di studi. Mi ritengo quindi pienamente soddisfatto del suo esito finale.

